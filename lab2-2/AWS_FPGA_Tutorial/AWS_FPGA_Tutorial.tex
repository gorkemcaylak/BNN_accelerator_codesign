\PassOptionsToPackage{unicode=true}{hyperref} % options for packages loaded elsewhere
\PassOptionsToPackage{hyphens}{url}
%
\documentclass[]{article}
\usepackage{lmodern}
\usepackage{amssymb,amsmath}
\usepackage{ifxetex,ifluatex}
\usepackage{fixltx2e} % provides \textsubscript
\ifnum 0\ifxetex 1\fi\ifluatex 1\fi=0 % if pdftex
  \usepackage[T1]{fontenc}
  \usepackage[utf8]{inputenc}
  \usepackage{textcomp} % provides euro and other symbols
\else % if luatex or xelatex
  \usepackage{unicode-math}
  \defaultfontfeatures{Ligatures=TeX,Scale=MatchLowercase}
\fi
% use upquote if available, for straight quotes in verbatim environments
\IfFileExists{upquote.sty}{\usepackage{upquote}}{}
% use microtype if available
\IfFileExists{microtype.sty}{%
\usepackage[]{microtype}
\UseMicrotypeSet[protrusion]{basicmath} % disable protrusion for tt fonts
}{}
\IfFileExists{parskip.sty}{%
\usepackage{parskip}
}{% else
\setlength{\parindent}{0pt}
\setlength{\parskip}{6pt plus 2pt minus 1pt}
}
\usepackage{hyperref}
\hypersetup{
            pdfborder={0 0 0},
            breaklinks=true}
\urlstyle{same}  % don't use monospace font for urls
\setlength{\emergencystretch}{3em}  % prevent overfull lines
\providecommand{\tightlist}{%
  \setlength{\itemsep}{0pt}\setlength{\parskip}{0pt}}
\setcounter{secnumdepth}{0}
% Redefines (sub)paragraphs to behave more like sections
\ifx\paragraph\undefined\else
\let\oldparagraph\paragraph
\renewcommand{\paragraph}[1]{\oldparagraph{#1}\mbox{}}
\fi
\ifx\subparagraph\undefined\else
\let\oldsubparagraph\subparagraph
\renewcommand{\subparagraph}[1]{\oldsubparagraph{#1}\mbox{}}
\fi


\usepackage{geometry}
\geometry{a4paper,scale=0.75}


% set default figure placement to htbp
\makeatletter
\def\fps@figure{htbp}
\makeatother


\date{}

\begin{document}

\hypertarget{header-n0}{%
\section{AWS FPGA INSTANCE CONFIGURATION GUIDE}\label{header-n0}}

By \textbf{Kyle Lu} based on the original version by \textbf{Joseph
Featherston}

Last Updated on \textbf{May 19th, 2018}

\tableofcontents

\hypertarget{header-n13}{%
\section{0. Preamble}\label{header-n13}}

Special credits go to XXX who created the original version of
\texttt{aws\_f1\_user\_guide\_v1}. It was a great guide for setting up
the AWS FPGA instances.

However when I was following this original guide to set up my first
instance, I found there were some places of ambiguity, omission and
minor mistakes, which cause me some detour.

To save you from the trouble and confusion, I created this guide based
on the original version, and tried my best to make it beginner-friendly.
Major changes includes:

\begin{itemize}
\item
  Added more detailed instructions 
\item
  Added explanation for those abbreviation that user might not be
  familiar with.
\item
  Corrected the minor mistakes
\item
  Eliminated the ambiguity and provided missing information
\item
  Language is written in a beginner's perspective to avoid unnecessary
  confusion.
\end{itemize}

Hope this guide gives you a concise and clear start with minimum time
and maximum efficacy.

Enjoy your journey!

\hypertarget{header-n42}{%
\section{1. AWS Account Setup}\label{header-n42}}

\hypertarget{header-n45}{%
\subsubsection{1.1 Creating a New AWS Account}\label{header-n45}}

\begin{itemize}
\item
  Create a new \textbf{AWS account} using
  \href{https://portal.aws.amazon.com/billing/signup\#/start}{this link}
\item
  You should use your @edu email address
\item
  You will need to add a credit card when creating an account
\end{itemize}

\hypertarget{header-n56}{%
\subsubsection{1.2 Apply for AWS Educate with \$100 Free
Credits}\label{header-n56}}

\begin{itemize}
\item
  Go to \href{https://aws.amazon.com/education/awseducate/}{this page}
  to apply for \textbf{AWS Educate for student}
\item
  Do not choose "AWS Educate Starter Account", you need a regular
  account instead of a "Starter Account" for full access to the
  resources
\item
  After filling out all the forms, you will receive an email with a
  promotion code of \textbf{\$100 credits}, this took about 10 minutes
  in my case
\item
  Apply the promotion code following the steps in the email
\end{itemize}

\hypertarget{header-n70}{%
\subsubsection{1.3 Setting up Billing Alarm}\label{header-n70}}

\begin{itemize}
\item
  The \textbf{AWS Console} is where you can manage basically everything
  of your AWS account, use \href{https://console.aws.amazon.com/}{this
  link} to go to your AWS Console
\item
  Navigate through
  \texttt{Services\ —\textgreater{}\ Billing\ —\textgreater{}\ Preferences},
  and then click on "Receive Billing Alerts"
\item
  At this Point, you should navigate through the AWS Console to get
  yourself familiar with its layout and setting
\end{itemize}

\hypertarget{header-n534}{%
\subsubsection{1.4 Plan Ahead: Increase the Instance
limit}\label{header-n534}}

\begin{itemize}
\item
  We will need an m4 instance and an F1 instance for this course,
  however our account type only allows us to use 1 instance. So we have
  to increase this limit before we can proceed
\item
  Use
  \href{https://console.aws.amazon.com/support/v1\#/case/create}{this
  link} to reach out to the customer service for raising the limit
\item
  Fill out the form and wait for AWS staff to process your request
\item
  Please be aware that this could take \textbf{2\textasciitilde{}3 days}
  for the request to be processed. So, \textbf{plan ahead} for your time
\end{itemize}

\hypertarget{header-n84}{%
\section{2. Launching Development Instance}\label{header-n84}}

\hypertarget{header-n85}{%
\subsubsection{2.1 Launching New Instance}\label{header-n85}}

\hypertarget{header-n86}{%
\paragraph{2.1.1 Find the Launch Entry}\label{header-n86}}

\begin{itemize}
\item
  Use \href{https://console.aws.amazon.com/}{this link} to open AWS
  Console
\item
  Navigate through \texttt{Services\ —\textgreater{}\ EC2} 
\item
  Check your zone on the top right, ensure you are in
  \texttt{US\ East\ (N.\ Virginia)}
\item
  Click on "\textbf{Launch Instance}"
\end{itemize}

\hypertarget{header-n100}{%
\paragraph{2.1.2 Choose Amazon Machine Image (AMI)}\label{header-n100}}

\begin{itemize}
\item
  Click on "\textbf{AWS Marketplace}" on the left, and enter
  "\textbf{FPGA}" in the search bar
\item
  Select \texttt{FPGA\ Developer\ AMI} to start configuration
\item
  Scan through the pop-up window and hit "\textbf{continue}"
\end{itemize}

\hypertarget{header-n111}{%
\paragraph{2.1.3 Choose Instance Type}\label{header-n111}}

\begin{itemize}
\item
  The lowest recommended instance is \texttt{t2.xlarge} (4CPU/16GB ram)
\item
  If you want to run multiple builds in parallel, you probably want one
  with more RAM, like \texttt{m4.2xlarge/m4.4xlarge}, in the course, we
  recommend to use \texttt{m4.2xlarge}
\item
  \textbf{A word on why choosing this instance}: In fact, only the f1
  series instances are actually connected to a real FPGA, however f1
  instances are expensive, given we only have \$100 credits so far.
  Therefore, we start with non-f1 instances and continue with f1 only
  when we finished all the pre-work
\item
  Hit "\textbf{Next: Configure Instance Details}", no need to change
  anything on this page
\item
  Hit "\textbf{Next: add Storage}"
\end{itemize}

\hypertarget{header-n128}{%
\paragraph{2.1.4 Add Storage}\label{header-n128}}

\begin{itemize}
\item
  There are two volumes shown on this page 
\item
  \textbf{The First Volume} is the 70GB root drive

  \begin{itemize}
  \item
    This is where OS and Vivado is installed. Do not modify this volume
    or save things here. This volume will be deleted once the instance
    is terminated
  \end{itemize}
\item
  \textbf{The Second Volume} is where you will store your work

  \begin{itemize}
  \item
    The size should be increased to 30 GB
  \item
    For your information, EBS refers tp "Elastic Blocking Store", you do
    not need to understand this though
  \end{itemize}
\item
  Hit "\textbf{Review and Launch}", since you don't need to change
  anything in the sections afterwards.
\item
  Review the summary and hit "\textbf{Launch}" (It is normal that there
  is a warning message saying "Your instance configuration is not
  eligible for the free usage tier")
\end{itemize}

\hypertarget{header-n156}{%
\paragraph{2.1.5 Create Key Pair}\label{header-n156}}

\begin{itemize}
\item
  There will be a pop-up window
\item
  If this is your first time,

  \begin{itemize}
  \item
    Select "\textbf{Create a new key pair}"
  \item
    Name it something permanent, you will need to reuse it afterwards
  \item
    Carefully download and store it in a safe place on your laptop(or
    PC), for example \texttt{.ssh/}
  \item
    Put down your \texttt{path\ to\ the\ key\ pair} somewhere handy for
    later use 
  \item
    Never let anyone else access this file
  \end{itemize}
\item
  If this is not your first time,

  \begin{itemize}
  \item
    you may want to use your existing key pair
  \end{itemize}
\item
  Hit "\textbf{Launch Instance}"
\end{itemize}

\hypertarget{header-n190}{%
\paragraph{2.1.6 Check Your Instance}\label{header-n190}}

\begin{itemize}
\item
  Go back to your EC2 Console (Services/EC2)
\item
  Click on "\textbf{Instances}" on the left, and you will see you
  instance showing up there
\item
  Select it and put down the \texttt{public\ DNS} aside with the your
  key pair path for later use
\end{itemize}

\hypertarget{header-n203}{%
\subsubsection{2.2 Creating AWS Credentials}\label{header-n203}}

\begin{itemize}
\item
  Go back to your AWS Console
\item
  Click on your name on the top right, then click on"\textbf{Your
  Securtiy Credentials}"
\item
  In the pop-up window, hit "\textbf{Get Started with IAM users}"
\item
  Hit "\textbf{Add user}", enter a name and select "\textbf{Programmatic
  Access}"
\item
  Add \texttt{AmazonEC2FullAccess} and \texttt{AmazonS3FullAccess}
  Permissions
\item
  Copy the \texttt{Secret\ Access\ key} and \texttt{Access\ Key\ ID} to
  a secure location,

  \begin{itemize}
  \item
    This is the only time you will be able to access the Secure Access
    Key if you lose it yßou will need to create a new user
  \item
    These keys will let someone create instances and cause things to be
    billed to your account so be somewhat careful with them
  \end{itemize}
\end{itemize}

\hypertarget{header-n232}{%
\subsubsection{2.3 Create S3 Bucket}\label{header-n232}}

The S3 bucket will be used later by the AWS SDAccel scripts to upload
your DCP to AWS for AFI generation which will be packaged into a tar
file.

\begin{itemize}
\item
  Open terminal on your laptop (or something alike on PC), ssh into your
  instance using the following command

\begin{verbatim}
$ ssh –i <path-to-key-pair> centos@<public dns address>
\end{verbatim}
\end{itemize}

\begin{itemize}
\item
  Install \texttt{AWS\ Command\ Line\ Interface\ (CLI)}, please follow
  \href{http://docs.aws.amazon.com/cli/latest/userguide/installing.html}{this
  tutorial}
\item
  Configure AWS

\begin{verbatim}
$ aws configure
\end{verbatim}
\end{itemize}

\begin{itemize}
\item
  When prompted, enter as follows:

\begin{verbatim}
AWS Access Key ID [None]: <Access Key ID from last step>
AWS Secret Access Key [None]: <Secure Access Key from last step>
Default region name [None]: us-east-1
Default output format [None]: json
\end{verbatim}
\end{itemize}

\begin{itemize}
\item
  Create bucket and relevant folders, the bucket name should be
  different from any other bucket.

\begin{verbatim}
# Create an S3 bucket 
$ aws s3 mb s3://<bucket-name> --region us-east-1
# Create folder for your tar files
$ aws s3 mb s3://<bucket-name>/<dcp-folder-name> 
# Create a temp file
$ touch FILES_GO_HERE.txt    
# Put the file in the S3 folder
$ aws s3 cp FILES_GO_HERE.txt s3://<bucket-name>/<dcp-folder-name>/  
\end{verbatim}

\begin{verbatim}
# Create a folder to keep your logs
$ aws s3 mb s3://<bucket-name>/<logs-folder-name>  
# Create a temp file
$ touch LOGS_FILES_GO_HERE.txt    
# Put the file in the S3 folder
$ aws s3 cp LOGS_FILES_GO_HERE.txt s3://<bucket-name>/<logs-folder-name>/  
\end{verbatim}
\end{itemize}

\hypertarget{header-n266}{%
\subsubsection{2.4 Initial Setup}\label{header-n266}}

\begin{itemize}
\item
  Check the "\textbf{Instance status}" on your EC2 console, proceed to
  next step when it finishes initialization

  \begin{itemize}
  \item
    This takes several minutes from the time it is launched 
  \end{itemize}
\item
  Open terminal on your laptop (or something alike on PC), ssh into your
  instance using the following command

\begin{verbatim}
$ ssh –i <path-to-key-pair> centos@<public dns address>
\end{verbatim}
\end{itemize}

\begin{itemize}
\item
  The 30GB volume is mounted here
  \texttt{\textasciitilde{}/src/project\_data}

\begin{verbatim}
$ cd ~/src/project_data
\end{verbatim}
\end{itemize}

\begin{itemize}
\item
  Clone \texttt{aws-fpga\ tools}

\begin{verbatim}
$ git clone https://github.com/aws/aws-fpga.git
\end{verbatim}
\end{itemize}

\begin{itemize}
\item
  Copy \texttt{lab\ 2-2} and \texttt{harness} here

  \begin{itemize}
  \item
    You can use either \texttt{scp} or \texttt{git\ clone} to do this.
  \end{itemize}
\item
  Run the \texttt{SDACCEL\ setup\ script}, this must be done from the
  aws-fpga root directory

\begin{verbatim}
$ cd aws-fpga
$ source sdaccel_setup.sh
$ source $XILINX_SDX/settings64.sh
\end{verbatim}
\end{itemize}

\hypertarget{header-n299}{%
\subsubsection{2.5 Run Hardware Synthesis}\label{header-n299}}

\begin{itemize}
\item
  Go to your working directory and find the \texttt{typedefs.h} file in
  src, make sure \texttt{TARGET\_DEVICE} is
  \texttt{xilinx:aws-vu9p-f1:4ddr-xpr-2pr:4.0}

\begin{verbatim}
$ cd ~/src/project_data/lab2-2/
$ cd src/host
$ vi typedefs.h
\end{verbatim}
\end{itemize}

\begin{itemize}
\item
  Run the hardware synthesis

  \begin{itemize}
  \item
    This takes about \textbf{4 hours}
  \item
    Do not disconnect from the instance while doing this
  \end{itemize}

\begin{verbatim}
$ make ocl OCL_TARGET=hw OCL_PLATFORM=$AWS_PLATFORM
\end{verbatim}
\end{itemize}

\begin{itemize}
\item
  If you run into problem while doing this, and you are using a fairly
  new laptop, try run the following commands, and then restart from
  Section 2.2

\begin{verbatim}
$ export LC_CTYPE=en_US.UTF-8
$ export LC_ALL=en_US.UTF-8
\end{verbatim}
\item
  You can add these lines to \texttt{/etc/environment} to fix this
  problem permanently

\begin{verbatim}
  LANG=en_US.utf-8
  LC_ALL=en_US.utf-8
\end{verbatim}
\end{itemize}

\hypertarget{header-n338}{%
\subsubsection{2.6 Generating Amazon FPGA Image
(AFI)}\label{header-n338}}

\begin{itemize}
\item
  Run \texttt{Configure\ AWS} again if you started a new terminal
  session

\begin{verbatim}
$ aws configure
AWS Access Key ID [None]: <Access Key ID from last step>
AWS Secret Access Key [None]: <Secure Access Key from last step>
Default region name [None]: us-east-1
Default output format [None]: json
\end{verbatim}
\end{itemize}

\begin{itemize}
\item
  Submit the job to create FPGA image by running

\begin{verbatim}
$ $SDACCEL_DIR/tools/create_sdaccel_afi.sh \
-xclbin= <input_xilinx_fpga_binary_xclbin_filename> \
-o=<output_aws_fpga_binary_awsxclbin_filename_root> \
-s3_bucket=<bucket-name> \
-s3_dcp_key=<dcp-folder-name> \
-s3_logs_key=<logs-folder-name>
\end{verbatim}
\end{itemize}

\begin{itemize}
\item
  You can check the status of the run by running the following command,
  it lists all your fpga images

\begin{verbatim}
$ aws ec2 describe-fpga-images --owners self
\end{verbatim}
\end{itemize}

\begin{itemize}
\item
  Alternatively, you can look up specific pga image by its
  \texttt{\textless{}AFI\ ID\textgreater{}}, which is shown in
  \texttt{\textless{}timestamp\textgreater{}\_afi\_id.txt} created when
  script runs

\begin{verbatim}
$ aws ec2 describe-fpga-images –fpga-image-ids <AFI ID>
\end{verbatim}
\end{itemize}

\begin{itemize}
\item
  You need to wait for ``\textbf{State}'' to change from
  "\textbf{pending}" into ``\textbf{available}'' before you can use the
  image

  \begin{itemize}
  \item
    This could take about \textbf{1 hour}
  \end{itemize}
\end{itemize}

\hypertarget{header-n373}{%
\subsubsection{2.7 Stopping the Instance}\label{header-n373}}

\begin{itemize}
\item
  Go back to AWS Console and navigate to the instances page
\item
  Stopping or termination instances

  \begin{itemize}
  \item
    You can stop or terminate the instance by right click on the
    instance and select corresponding action
  \item
    You can resume stopped instances while terminated instances are gone
  \end{itemize}
\item
  Charges

  \begin{itemize}
  \item
    You are not charged for an instance in stopped or terminated state
  \item
    You are charged for instances in the running state, volumes and
    snapshots(i.e. backups) so you should check that there isn't
    anything there you don't need
  \item
    You do need to pay for the 70GB whenever the instance is in the
    stopped state The price of EBS is \textasciitilde{}\$0.10/GB-month,
    so the volume will cost \$7/month, while it is probably worth
    spending the \$3/month to keep your work volume around if you are
    using it relatively frequently
  \end{itemize}
\end{itemize}

\hypertarget{header-n404}{%
\section{3. Running on Hardware}\label{header-n404}}

\hypertarget{header-n419}{%
\subsubsection{3.1 Launch F1 Instance}\label{header-n419}}

\begin{itemize}
\item
  The process is the same with \texttt{Section\ 2.1}, except the
  following:
\item
  At "Step 2: Choose an Instance Type": choose \texttt{f1.2xlarge}
  instead of previous instance type

  \begin{itemize}
  \item
    These instances cost \$1.65/hour so try to limit how long you have
    them running
  \end{itemize}
\item
  At "Step 3: Configure Instance Details": in the subset dropdown, make
  sure you select the same availability zone as the previous 30GB volume

  \begin{itemize}
  \item
    You can check the availability zone of the previous 30GB volume from
    \texttt{EC2\ Console\ —\textgreater{}\ Volume}
  \end{itemize}
\item
  At "Step 4: Add Storage": we are going to use the previous 30GB volume
  we have created, therefore we don't need the second 5GB volume that
  added by default. So remove the second volume.
\end{itemize}

\hypertarget{header-n441}{%
\subsubsection{3.2 Mount Previous Volume}\label{header-n441}}

\begin{itemize}
\item
  Navigate to to volume page
  \texttt{EC2\ Console\ —\textgreater{}\ Volume}
\item
  Right click your 30GB volume, click on "\textbf{Detach}"
\item
  Go back to your instances and figure out the id of you f1.2 instance (
  you don't need to memorize the id, you only need to be able to tell
  which id is of your f1.2 instance, which one is of you m4.2 instance)
\item
  Right click your 30GB volume, click on "\textbf{Attach}", then select
  the id of your f1 instance
\item
  Open terminal on your laptop (or something alike on PC), ssh into your
  f1 instance using the following command with correct
  \texttt{public\ dns\ address}

\begin{verbatim}
$ ssh –i <path-to-key-pair> centos@<public dns address>
\end{verbatim}
\item
  Run the following command to see the block devices, find the one which
  is 30GB, put down the \texttt{path\ of\ your\ block}, which in my case
  is \texttt{/dev/xvdf}

\begin{verbatim}
$ lsblk
\end{verbatim}
\item
  We would like to mount our previous 30GB volume to the same directory
  with before for consistency, which is
  \texttt{\textasciitilde{}/src/project\_data}, however the AWS f1
  instance won't let us do it for whatever reason it is. So we are using
  the following steps to work it around

  \begin{itemize}
  \item
    Go to \texttt{\textasciitilde{}/src/} and create a new directory
    named \texttt{project}

\begin{verbatim}
$ cd ~/src
$ mkdir project
\end{verbatim}
  \end{itemize}
\end{itemize}

\begin{itemize}
\item
  Mount the 30GB volume to \texttt{\textasciitilde{}/src/project}

\begin{verbatim}
$ sudo mount /dev/xvdf ~/src/project
\end{verbatim}
\item
  Copy the fold \texttt{aws-fpga} from
  \texttt{\textasciitilde{}/src/project} to
  \texttt{\textasciitilde{}/src/project\_data}

\begin{verbatim}
$ cp -r aws-fpga/ ../project_data/
\end{verbatim}
\end{itemize}

\begin{itemize}
\item
  Note that we only need fold \texttt{aws-fpga} to be in directory
  \texttt{\textasciitilde{}/src/project\_data}, because some inner codes
  of \texttt{aws-fpga} are hard-coded and bonded up to this directory
  (not our fault), we can surely change the inner code, however it is to
  much work to do so, therefore we are doing this as a workaround.
\end{itemize}

\hypertarget{header-n486}{%
\subsubsection{3.3 Run Instance}\label{header-n486}}

\begin{itemize}
\item
  Run SDAccel Setup if you haven't since rebooting

\begin{verbatim}
$ source ~/src/project_data/aws-fpga/sdaccel_setup.sh
\end{verbatim}
\item
  Switch to root shell

\begin{verbatim}
$ sudo sh
$ source /opt/Xilinx/SDx/2017.1.rte.4ddr/setup.sh
\end{verbatim}
\item
  Go to your working directory and run the application following the
  instruction in \texttt{README.md}, which is shown below in our case
  \texttt{lab2-2}

\begin{verbatim}
$ cd /home/centos/src/project/lab2-2
$ ./cordic_host.exe -f <path_to_xclbin_file>
\end{verbatim}
\end{itemize}

\end{document}
